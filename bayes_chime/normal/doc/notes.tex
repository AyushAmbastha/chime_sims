\batchmode
\documentclass[paper=a4, fontsize=12pt, prl, notitlepage]{revtex4-1}
\usepackage[ssp]{xetex_import}
\linespread{1.4}
\errorstopmode

\begin{document}

\title{Math behind Bayesian uncertainty propagation using normal approximations}
\author{Christopher Körber}
\affiliation{%
Department of Physics,
University of California,
Berkeley, CA 94720, USA\\
Nuclear Science Division, %
LBNL,
Berkeley, CA 94720, USA
}
\date{\today}
\begin{abstract}
This document addresses the math used to propagate errors and to infer posterior distributions (Bayesian statistics) used in \href{https://github.com/pennsignals/chime_sims/pull/49}{the performance boost through normal approximations PR in the \texttt{pennsignals/chime\_sims} repo}.
\end{abstract}

\maketitle

%==============================================================================
% Content
%==============================================================================

\section{Assumptions}
The general theme of this document is: \textit{casting everything to normal distribution simplifies computations}.

In general, the techniques in PR \#49 make two approximations
\begin{enumerate}
    \item All model and data \textit{input} distributions can be sufficiently approximated with multivariate normal distributions
    \item The posterior distributions can be sufficiently approximated by a multivariate normal distribution
\end{enumerate}

\section{Error propagation}
The \texttt{gvar} module is used to analytically propagate errors.
The eqution used to computed the standard deviation $\sigma_f$ of a function $f(\vec p)$ at a given point in the parameter space $\bar {\vec p}$ is
\begin{equation}
    \sigma_f^2(\bar {\vec p})
    =
    \left[
        \sum_{i,j=1}^N
        \frac{\partial f(\vec p)}{\partial p_i}
        \left(\Sigma_p\right)_{ij}
        \frac{\partial f(\vec p)}{\partial p_j}
    \right]_{\vec p = \bar {\vec p}}
    \, ,
\end{equation}
where $\Sigma_p$ is the covariance matrix of the normally distributed parameters $\vec p$.
In the uncorrelated case, this simplifies to
\begin{equation}
    \sigma_f^2(\bar {\vec p})
    =
    \left[
        \sum_{i=1}^N
        \left(\frac{\partial f(\vec p)}{\partial p_i}\right)^2
        \sigma_{p_i}^2
    \right]_{\vec p = \bar {\vec p}}
    \, .
\end{equation}

The module \texttt{gvar} is capable of tracing such derivatives since implemented functions are aware of their analytical derivatives.

\subsubsection{Example}
For example, suppose
\begin{align}
    f(x, y)
    &=
    x y^2
    && \Rightarrow &
    \sigma_f^2(\bar{\vec p})
    &=
    y^4 (\Sigma_p)_{xx}
    + 2 x y^3  \left[ (\Sigma_p)_{xy} + (\Sigma_p)_{yx} \right]
    + 4 x^2 y^2 (\Sigma_p)_{yy}
\end{align}
Thus, if $x, y$ follow a multivariate normal distribution of mean $\bar{\vec p}$ and covariance $\Sigma_p$, one finds
\begin{align}
    \bar{\vec p}
    &=
    \begin{pmatrix}
        1 \\ -2
    \end{pmatrix}
    \, , &
    \Sigma_p
    &=
    \begin{pmatrix}
        2 & 1 \\ 1 & 1
    \end{pmatrix}
    &
    \Rightarrow
    \sigma_f(\bar{\vec p})
    &=
    \sqrt{32 - 32 + 16} = 4
\end{align}
The corresponding \texttt{gvar} code returns
\begin{lstlisting}[style=python]
from gvar import gvar

mean = [1, -2]
cov = [[2, 1], [1, 1]]
x, y = gvar(mean, cov)
f = x * y ** 2
print(f)

> 4.0(4.0)
\end{lstlisting}


\section{Computation of the posterior}

This section explains how the \texttt{lsqfit} module approximates the posterior distribution $P(\vec p|D,M)$ given data $D$, an input model $M$ and it's corresponding priors $P(\vec p| M)$.

\subsection{Defintions}

The posterior distribution $P(\vec p|D, M)$ is proportional to the prior times the probability of the data given the model and parameters $P(D|\vec p, M)$ (the likelihood)
\begin{align}
    P(\vec p|D, M) &=
    \frac{P(D|\vec p, M)P(\vec p | M)}{P(D|M)}
    \propto
    P(D|\vec p, M)P(\vec p | M)
    \, .
\end{align}
The marginal likelihood of the data $D$ given the model $M$ is obtained by integrating over the whole parameter space
\begin{equation}
    P(D|M)
    =
    \int d \vec p P(D|\vec p, M)P(\vec p | M) \, .
\end{equation}
Because the posterior is normalized by the ratio of both distributions, one can neglect constant factors in the computation.

The likelihood of the data given the model and parameters is described by a $\chi^2$ distribution
\begin{equation}
    P(D|\vec p, M)
    \sim
    \exp\left\{
        - \frac{1}{2}
        \sum_{i,j=1}^N
        \left[y_i - \vec f_M(x_i, \vec p)\right]
        \left(\Sigma_y^{-1}\right)_{ij}
        \left[y_j - \vec f_M(x_j, \vec p)\right]
    \right\}
    =
    \exp\left\{
        - \frac{1}{2}
        \chi^2_D(\vec p)
    \right\}
    \, ,
\end{equation}
where $\Sigma_y$ is the covariance matrix of the data and $f_M(x_j, \vec p)$ the model function evaluated at point $x_j$ which aims to describe the data point $y_j$.

Maximizing the Likelihood as a function of $\vec p$ corresponds to minimizing the exponent--which is the standard $\chi^2$-minimization procedure.
Computing the posterior distribution function for a given prior $P(\vec p| M)$ is called Bayesian inference.
Normal approximations of the posterior distribution are somewhere in the middle of a full Bayesian treatment and regular $\chi^2$-minimization.

\subsection{Normal approximation of the posterior}
Including the multivariate normal prior distribution with mean $\vec p_0$ and covariance $\Sigma_{p_0}$, the posterior distribution is proportional to
\begin{align}
    P(\vec p|D, M)
    &\sim
    \exp\left\{
        - \frac{1}{2}
        \chi^2_D(\vec p)
        - \frac{1}{2}
        \chi^2_M(\vec p)
    \right\}
    =
    \exp\left\{
        - \frac{1}{2}
        \chi^2_{DM}(\vec p)
    \right\}
    \, , &
    \chi^2_M(\vec p)
    &=
    \left[\vec p - \vec p_0\right]^T \cdot
    \Sigma_{p_0}^{-1}
    \left[\vec p - \vec p_0\right]\, .
\end{align}
In short, the \texttt{lsqfit} module approximates the posterior by expressing the function $\chi^2_{DM}(\vec p)$ up to second order in $\vec p$ at the point $\bar{\vec p}$ where the first derivative vanishes (stationary or almost always minimal point)
\begin{align}
    \chi^2_{DM}(\vec p)
    & \approx
    \chi^2_{DM}(\bar{\vec p})
    +
    \left[\vec p - \bar{\vec p}\right]^T
    \Sigma_{DM}^{-1}(\bar{\vec p})
    \left[\vec p - \bar{\vec p}\right]^T
    \, , & &
    \left.\frac{\partial \chi^2_{DM}(\vec p)}{\partial p_\alpha}\right|_{\vec p = \bar{\vec p}} = 0 \, \quad \forall_\alpha \, .
\end{align}
In this approximation, the posterior is again a multivariate normal distribution of mean $\bar{\vec p}$ (same as maximal likelihood estimation) with covariance $\Sigma_{DM}(\bar{\vec p})$.
The \texttt{nonlinear\_fit} method numerically computes the vector which minimizes the posterior $\bar{\vec p}$ (fitting) and analytically computes and evaluates the covariance matrix $\Sigma_{DM}(\bar{\vec p})$ at this point.
The appendix A of \cite{Bouchard:2014ypa} describes how $\Sigma_{DM}(\bar{\vec p})$ is estimated using derivatives of the residuals with respect of prior parameters.
%
% \subsubsection{Example}
%
% Suppose the fit function is a linear function $f(x_i, \vec p) = p^{(0)} + p^{(1)} x_i$.
% In this case, the normal approximation of the posterior is exact
% \begin{align}
%     \chi^2_{DM}(\vec p)
%     & =
%     \sum_{i,j=1}^N
%     \left[y_i - p^{(0)} - p^{(1)} x_i \right]
%     \left(\Sigma_y^{-1}\right)_{ij}
%     \left[y_j  - p^{(0)} - p^{(1)} x_j\right]
%     +
%     \left[\begin{pmatrix} p^{(0)} \\ p^{(1)} \end{pmatrix} - \vec p_0 \right]^T
%     \left(\Sigma_p^{-1}\right)_{ij}
%     \left[\begin{pmatrix} p^{(0)} \\ p^{(1)} \end{pmatrix} - \vec p_0 \right]
%     \\
%     & =
%     \left[
%         \sum_{i,j=1}^N y_i \left(\Sigma_y^{-1}\right)_{ij} y_j
%         + \vec p_0^T
%         \left(\Sigma_p^{-1}\right)
%         \vec p_0
%     \right]
%     +
%     \left[\begin{pmatrix} p^{(0)} \\ p^{(1)} \end{pmatrix} - \bar{\vec p} \right]
%     \left(\Sigma_{DM}^{-1}\right)_{ij}
%     \left[\begin{pmatrix} p^{(0)} \\ p^{(1)} \end{pmatrix} - \bar{\vec p} \right]^T
%     \, ,
% \end{align}
% with
% \begin{align}
%     \bar{\vec p} &= \ldots
%     \\
%     \Sigma_{DM}^{-1} &= \ldots
% \end{align}

%==============================================================================
% End Content
%==============================================================================
\bibliography{notes.bib}{}
\bibliographystyle{plainurl}

\batchmode
\end{document}
