\batchmode
\documentclass[paper=a4, fontsize=12pt, prl, notitlepage]{revtex4-1}
\usepackage[ssp]{xetex_import}
\linespread{1.4}
\errorstopmode

\begin{document}

\title{Math behind Bayesian uncertainty propagation using normal approximations}
\author{Christopher Körber}
\affiliation{%
Department of Physics,
University of California,
Berkeley, CA 94720, USA\\
Nuclear Science Division, %
LBNL,
Berkeley, CA 94720, USA
}
\date{\today}
\begin{abstract}
This document addresses the math used to infer posterior distributions and propagate erros used in \href{https://github.com/pennsignals/chime_sims/pull/49}{the performance boost through normal approximations PR}.
\end{abstract}

\maketitle

%==============================================================================
% Content
%==============================================================================



\section{Assumptions}
The general theme of this document is: \textit{casting everything to normal distribution simplifies computations}.

In general, the techniques in PR \#49 make two approximations
\begin{enumerate}
    \item All underlaying \textit{input} distributions of the model can be sufficiently approximated with correlated normal distributions (this includes model parameters as well as data)
    \item The posterior distribution can be sufficiently approximated by a normal distribution
\end{enumerate}

\section{Error propagation}
The \texttt{gvar} module is used to analytically propagate errors.
The eqution used to computed the standard deviation of a function $f(\vec p)$ at a given point in the parameter space $\bar {\vec p}$ is
\begin{equation}
    \sigma_f^2(\bar {\vec p})
    =
    \left[
        \sum_{i,j=1}^N
        \frac{\partial f(\vec p)}{\partial p_i}
        \left(\Sigma_p\right)_{ij}
        \frac{\partial f(\vec p)}{\partial p_j}
    \right]_{\vec p = \bar {\vec p}}
    \, ,
\end{equation}
where $\Sigma_p$ is the covariance matrix of the normally distributed parameters $\vec p$.
In the uncorrelated case, this simplifies to
\begin{equation}
    \sigma_f^2(\bar {\vec p})
    =
    \left[
        \sum_{i=1}^N
        \left(\frac{\partial f(\vec p)}{\partial p_i}\right)^2
        \sigma_{p_i}^2
    \right]_{\vec p = \bar {\vec p}}
    \, .
\end{equation}

The module \texttt{gvar} is capable of tracing such derivatives because implelentated functions are aware of their analytical derivative.
E.g., if you add two \texttt{gvar}s, the resulting \texttt{gvar} knows that it's respective derivative in direction of the previous \texttt{gvar}s is one.




\section{Computation of the posterior}

This section explains how the \texttt{lsqfit} module approximates the posterior distribution $P(\vec p|D,M)$ given data $D$, an input model $M$ and it's corresponding priors $P(\vec p| M)$.

\subsection{Defintions}

The posterior distribution is proportional to the probability of the data given the model and parameters $P(D|\vec p, M)$ (the likelihood) times the prior of parameters
\begin{align}
    P(\vec p|D, M) &=
    \frac{P(D|\vec p, M)P(\vec p | M)}{P(D|M)}
    \propto
    P(D|\vec p, M)P(\vec p | M)
    \, .
\end{align}
The marginal likelihood of the data $D$ given the model $M$ is obtained by integrating over the whole parameter space
\begin{equation}
	P(D|M)
	=
	\int d \vec p P(D|\vec p, M)P(\vec p | M) \, .
\end{equation}
Because we effectively normalize the posterior by the ratio of both distributions, we can neglect constant factors in the computation.

The likelihood of the data given the model and parameters is described by a $\chi^2$ distribution
\begin{equation}
    P(D|\vec p, M)
    \sim
    \exp\left\{
        - \frac{1}{2}
        \sum_{i,j=1}^N
        \left[y_i - \vec f_M(x_i, \vec p)\right]
        \left(\Sigma_y^{-1}\right)_{ij}
        \left[y_j - \vec f_M(x_j, \vec p)\right]
    \right\}
    =
    \exp\left\{
        - \frac{1}{2}
        \chi^2_D(\vec p)
    \right\}
    \, ,
\end{equation}
where $\Sigma_y$ is the covariance matrix of the data and $f_M(x_j, \vec p)$ the model function evaluated at point $x_j$ which aims to describe the data point $y_j$.

Maximizing the Likelihood as a function of $\vec p$ corresponds to minimizing the exponent which is the standard $\chi^2$-minimization procedure.
Computing the posterior distribution function for a given prior $P(\vec p|D, M)$ is what we call Bayesian inference.
Normal approximations of the posterior distribution are somewhere in the middle of a full Bayesian treatment and regular $\chi^2$-minimization.

\subsection{Normal approximation of the posterior}
The including the multivariate normal prior distribution with mean $\vec p_0$ and covariance $\Sigma_{p_0}$, posterior distribution is proportional to
\begin{align}
    P(\vec p|D, M)
    &\sim
    \exp\left\{
        - \frac{1}{2}
        \chi^2_D(\vec p)
        - \frac{1}{2}
        \chi^2_M(\vec p)
    \right\}
    =
    \exp\left\{
        - \frac{1}{2}
        \chi^2_{DM}(\vec p)
    \right\}
    \, , &
    \chi^2_M(\vec p)
    &=
    \left[\vec p - \vec p_0\right]^T \cdot
    \Sigma_{p_0}^{-1}
    \left[\vec p - \vec p_0\right]\, .
\end{align}
In short, the \texttt{lsqfit} module approximates the posterior by evaluating the combined prior and data $\chi^2_{DM}(\vec p)$ function up to second order in the parameter space at the point which minimizes the $\chi^2_{DM}(\vec p)$
\begin{align}
    \chi^2_{DM}(\vec p)
    & \approx
    \chi^2_{DM}(\bar{\vec p})
    +
    \left[\vec p - \bar{\vec p}\right]^T
    \Sigma_{DM}^{-1}(\bar{\vec p})
    \left[\vec p - \bar{\vec p}\right]^T
    \, , & &
    \left.\frac{\partial \chi^2_{DM}(\vec p)}{\partial p_\alpha}\right|_{\vec p = \bar{\vec p}} = 0 \, \quad \forall_\alpha \, .
\end{align}
In this approximation, the posterior is again a normal distribution of mean $\bar{\vec p}$ (same as maximal likelihood estimation) with covariance $\Sigma_{DM}(\bar{\vec p})$.
The vector $\bar{\vec p}$ and covarinace $\Sigma_{DM}(\bar{\vec p})$ are computed by the \texttt{nonlinear\_fit} method (the estimation of $\bar{\vec p}$ is the \textit{fitting} process).
The appendix A of \cite{Bouchard:2014ypa} describes how $\Sigma_{DM}(\bar{\vec p})$ is estimated using derivatives of the residulas with respect of prior parameters.



%==============================================================================
% End Content
%==============================================================================
\bibliography{notes.bib}{}
\bibliographystyle{plainurl}

\batchmode
\end{document}
